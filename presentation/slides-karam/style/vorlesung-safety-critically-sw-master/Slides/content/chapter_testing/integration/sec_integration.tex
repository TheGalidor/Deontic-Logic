\section{Integration Testing}

\begin{frame}{Integration Testing}
  \begin{itemize}
    \item Should one test each module independently and then combine them all,
or should one combine the next module to be tested with the set of previously tested modules before it is tested?
    \item This is called \alert{incremental} vs. \alert{non-incremental} testing.
  \end{itemize}
\end{frame}

\begin{Frame}{Non-incremental Testing}
  \begin{itemize}
    \item Testing a module $A$ requires\newline
      \ \ -- a \alert{driver module} for feeding inputs to $A$ and \newline
      \ \ -- \alert{Stub modules} replacing the modules called by $A$.
    \item Stubs are \alert{models} of the modules they represent.
  \end{itemize}
\end{Frame}

\begin{Frame}{Incremental Testing}
  \begin{itemize}
    \item Modules can call other modules.
    \item If mutually recursive modules are grouped as units, the interaction
    of modules can be represented as a \alert{directed acyclic graph}.
  \end{itemize}

  \xxx

  \inhead{Bottom-up Testing}
  \begin{itemize}
    \item Start testing the modules that do not call any others.
    \item Test a module only after all the modules it calls have been tested.
  \end{itemize}

  \xxx

  \inhead{Top-down Testing}
  \begin{itemize}
    \item Start testing the top module.
    \item Test a module after replacing the modules it calls by stubs (no drivers needed).
  \end{itemize}
\end{Frame}

\begin{Frame}{Non-incremental Compared to\\ Incremental Testing}
  \inhead{Advantages}
  \begin{itemize}
    \item[\goodmark] More opportunities for parallel activities.
    \item[\goodmark] Less CPU-time (only small parts of the system are tested).
  \end{itemize}

  \xxx

  \inhead{Disadvantages}
  \begin{itemize}
    \item[\badmark] Requires to program stubs and drivers.
    \item[\badmark] Errors related to interface problems are caught very late.
  \end{itemize}
\end{Frame}

\begin{Frame}{Top-down Compared To\\ Bottom-up Incremental Testing}
  \inhead{Advantages}
  \begin{itemize}
    \item[\goodmark] No need for drivers.
    \item[\goodmark] Good for errors \enquote{near the top} of the program.
    \item[\goodmark] Test cases are easier to represent.
    \item[\goodmark] Allows for prototype demonstrations (early skeletal program).
  \end{itemize}

  \xxx

  \inhead{Disadvantages}
  \begin{itemize}
    \item[\badmark] Requires stubs.
    \item[\badmark] Stubs may make testing some paths impossible, or may introduce \enquote{spurious} bugs.
    \item[\badmark] Bad for errors \enquote{near the bottom} of the program.
  \end{itemize}
\end{Frame}