\RequirePackage{etoolbox}
\providetoggle{isSolution}

% Show or hide solutions, according to `isSolution` flag set before \documentclass
\iftoggle{isSolution} {\PassOptionsToClass{solution}{exercise}} {}
% You can also use \exerciseHideSolutions to explicitely hide them.

\documentclass[language=en,sheet=6,prefix]{exercise}

\exerciseCourse{Development of Safety-Critical Systems (Sichere Software)}
\exerciseAuthors{Martin Leucker, Karam Kharraz, Felix Lange}
\exerciseTerm{Winter Term 2018}
\exerciseDeadline{}
\exerciseDate{December 8, 2018}

% Show or hide solutions, according to `isSolution` flag set before \documentclass
% \exerciseShowSolutions
% You can also use \exerciseHideSolutions to explicitely hide them.

\lstdefinelanguage[]{Coq}[]{}{literate=%
{>}{{$>$}}{1}
{<}{{$<$}}{1}
{=}{{$=$}}{1}
{-}{{$-$}}{1}
{+}{{$+$ }}{1}
{~}{{$\sim$}}{1}
{===}{$\equiv$ }{1}
{...}{$\ldots$ }{2}
{<--}{\reflectbox{$\leadsto$}}{3}
{->}{$\rightarrow$ }{1}
{=>}{$\Rightarrow$ }{1}
{>=}{$\ge$}{1}
{<=}{$\le$}{1}
{:=}{$\coloneqq$}{2}
{/\\}{{$\wedge$}}{1}
{\\/}{{$\vee$}}{1}
{forall}{$\forall$}{1}
{TTrue}{{True}}{4}
{FFalse}{{False}}{5}
{True}{{$\top$}}{1}
{False}{{$\bot$}}{1}
%{'a}{{$\alpha$}}{1}
%{'b}{{$\beta$}}{1}
%{'c}{{$\gamma$}}{1}
%{'d}{{$\delta$}}{1}
%{'e}{{$\varepsilon$}}{1}}
,
morekeywords={Inductive,Definition,Fixpoint,Lemma,Theorem,Proof,Qed,Require,Import,Set,Type,Prop,match,fun,apply,intros,exact,destruct,split,simpl,reflexivity,exfalso,induction,rewrite,left,right}
}
\lstdefinestyle{Coq}{language=Coq}
\lstset{style=Coq}

\usepackage{utf8-math}
\usepackage{xspace}
\usepackage{tikz}
\usepackage{dirtytalk} % \say command
\usepackage{enumitem} % customize \enumerate
\usepackage{mathpartir} % Inference rules
\usepackage{lscape} % allow landscape
\usetikzlibrary{arrows,automata}


\newcommand{\ltl}{LTL}
\renewcommand{\phi}{\varphi}

% Temporal operators
\newcommand{\U}{\ensuremath{\operatorname{U}}\xspace}
\newcommand{\XU}{\ensuremath{\operatorname{XU}}\xspace}
%\newcommand{\R}{\ensuremath{\operatorname{R}}\xspace}
\newcommand{\X}{\ensuremath{\operatorname{X}}\xspace}
\newcommand{\WX}{\ensuremath{\operatorname{\bar{X}}}\xspace}
\newcommand{\G}{\ensuremath{\operatorname{G}}\xspace}
\newcommand{\F}{\ensuremath{\operatorname{F}}\xspace}
\newcommand{\W}{\ensuremath{\operatorname{W}}\xspace}

\begin{document}

\task[Propositional Logic]

Recall the proof system of natural deduction defined by the four inference rules. Here is a proof that inconsistent assumptions $p$ and $¬p$ imply any $q$.
%
$$
\inferrule*[Right=ImpIntro]
{
	\inferrule*[Right=ImpIntro]
	{
		\inferrule*[Right=DoubleNeg]
		{
      \inferrule*[Right=ImpElim]
  		{
        \inferrule*[left={\footnotesize$p → ⊥ ∈ \{ ... \}$},Right=Assump,vcenter=true,rightskip=-3em]
        { }
        {p, p → ⊥, q → ⊥ ⊢ p → ⊥}\\
				\inferrule*[left={\footnotesize$p ∈ \{ ... \}$},Right=Assump,vcenter=true]
        { }
        {p, p → ⊥, q → ⊥ ⊢ p}\\
      }
  		{p, p → ⊥, q → ⊥ ⊢ ⊥}
    }
		{p, p → ⊥ ⊢ q}
	}
	{p ⊢ (p → ⊥) → q}
}
{⊢ p → (p → ⊥) → q}
$$

\begin{enumerate}
\item
Use the rules to construct proof trees for the following formulas. Prove them, and also $p → (p → ⊥) → q$, in Coq using tactics. Finally, prove them by giving explicit proof terms. Use the given template file for the Coq proofs.
%
\begin{enumerate}[label=(\roman*)]
	\item $p → (p → ⊥) → ⊥$
  \item $(p → q) → (q → r) → (p → r)$
\end{enumerate}

\item
Extend the syntax of propositional logic by conjunction and disjunction. Extend the semantics (evaluation function). Extend the inference rules of the proof system. \textit{Hint:} For both operators, you will need introduction and elimination rule(s), analogously to the two rules \textsc{ImpIntro}, \textsc{ImpElim}. If you get stuck, you may get inspiration from the internet.

\item
Prove the following formulas in your extended proof system. Then prove them in Coq both by using tactics and by giving a proof term (use again the template file).\\[-2em]

\begin{minipage}[t]{17em}
  \begin{enumerate}[label=(\roman*)]
    \item $(p ∧ q) → (q ∧ p)$
    \item $p ∧ (q ∨ r) → (p ∧ q) ∨ (p ∧ r)$
  \end{enumerate}
\end{minipage}
\begin{minipage}[t]{20em}
\begin{enumerate}[label={}]
  \item \textit{\say{$∧$ is commutative}}
  \item \textit{\say{$∧$ is distributive over $∨$}}
\end{enumerate}
\end{minipage}

\end{enumerate}

\begin{solution}
\begin{landscape}
\begin{enumerate}
\item
\begin{enumerate}[label=(\roman*)]
\item
$$
\inferrule*[Right=ImpIntro]
{
	\inferrule*[Right=ImpIntro]
	{
    \inferrule*[Right=ImpElim]
		{
      \inferrule*[left={\footnotesize$p → ⊥ ∈ \{ ... \}$},Right=Assump,vcenter=true,rightskip=-3em]
      { }
      {p, p → ⊥ ⊢ p → ⊥}\\
			\inferrule*[left={\footnotesize$p ∈ \{ ... \}$},Right=Assump,vcenter=true]
      { }
      {p, p → ⊥ ⊢ p}\\
    }
		{p, p → ⊥ ⊢ ⊥}
	}
	{p ⊢ (p → ⊥) → ⊥}
}
{⊢ p → (p → ⊥) → ⊥}
$$

\item
$$
\inferrule*[Right=ImpIntro]
{
	\inferrule*[Right=ImpIntro]
	{
		\inferrule*[Right=ImpIntro]
		{
			\inferrule*[Right=ImpElim]
			{
				\inferrule*[left={\footnotesize$q → r ∈ \{ ... \}$},Right=Assump,vcenter=true,rightskip=-3em]
				{ }
				{p → q, q → r, p ⊢ q → r}\\
				\inferrule*[Right=ImpElim]
				{
					\inferrule*[left={\footnotesize$p → q ∈ \{ ... \}$},Right=Assump,vcenter=true,rightskip=-3em]
					{ }
					{p → q, q → r, p ⊢ p → q}\\
					\inferrule*[left={\footnotesize$p ∈ \{ ... \}$},Right=Assump,vcenter=true,rightskip=-3em]
					{ }
					{p → q, q → r, p ⊢ p}
				}
				{p → q, q → r, p ⊢ q}
			}
			{p → q, q → r, p ⊢ r}
		}
		{p → q, q → r ⊢ p → r}
	}
	{p → q ⊢ (q → r) → (p → r)}
}
{⊢ (p → q) → (q → r) → (p → r)}
$$
\end{enumerate}

\item
Extension of the evaluation function:
\begin{align*}
\llbracket φ_1 ∧ φ_2 \rrbracket_v &:= \left\{\begin{aligned}
	1 & \quad\text{if $\llbracket φ_1 \rrbracket_v = 1$ and $\llbracket φ_2 \rrbracket_v = 1$}\\
	0 & \quad\text{otherwise}
\end{aligned}\right.\\
\llbracket φ_1 ∨ φ_2 \rrbracket_v &:= \left\{\begin{aligned}
	1 & \quad\text{if $\llbracket φ_1 \rrbracket_v = 1$ or $\llbracket φ_2 \rrbracket_v = 1$}\\
	0 & \quad\text{otherwise}
\end{aligned}\right.
\end{align*}

Extension of the inference rules:
$$
\inferrule*[right=ConjIntro]
{
	Γ ⊢ φ_1\\
	Γ ⊢ φ_2
}
{
	Γ ⊢ φ_1 ∧ φ_2
}
%
\qquad
%
\inferrule*[right=ConjElimL]
{
	Γ ⊢ φ_1 ∧ φ_2
}
{
	Γ ⊢ φ_1
}
%
\qquad
%
\inferrule*[right=ConjElimR]
{
	Γ ⊢ φ_1 ∧ φ_2
}
{
	Γ ⊢ φ_2
}
$$
%
~\\[1em]
%
$$
\inferrule*[right=DisjIntroL]
{
	Γ ⊢ φ_1
}
{
	Γ ⊢ φ_1 ∨ φ_2
}
%
\qquad
%
\inferrule*[right=DisjIntroR]
{
	Γ ⊢ φ_2
}
{
	Γ ⊢ φ_1 ∨ φ_2
}
%
\qquad
%
\inferrule*[right=DisjElim]
{
	Γ ⊢ φ_1 ∨ φ_2\\
	Γ ⊢ φ_1 → ψ\\
	Γ ⊢ φ_2 → ψ
}
{
	Γ ⊢ ψ
}
$$

\item
$$
\inferrule*[Right=ImpIntro]
{
	\inferrule*[Right=ConjIntro]
	{
		\inferrule*[Left=ConjElimR]
		{
			\inferrule*[left={\footnotesize$p ∧ q ∈ \{ ... \}$},Right=Assump,vcenter=true,rightskip=-6em]
			{ }
			{p ∧ q ⊢ p ∧ q}
		}
		{p ∧ q ⊢ q}\\
		\inferrule*[Right=ConjElimL]
		{
			\inferrule*[left={\footnotesize$p ∧ q ∈ \{ ... \}$},Right=Assump,vcenter=true,rightskip=-6em]
			{ }
			{p ∧ q ⊢ p ∧ q}
		}
		{p ∧ q ⊢ p}
	}
	{p ∧ q ⊢ q ∧ p}
}
{⊢ (p ∧ q) → (q ∧ p)}
$$
%
~\\[1em]
%
$$
\inferrule*[Right=ImpIntro]
{
	\inferrule*[Right=DisjElim]
	{
		\inferrule*[Left=ConjElimR]
		{
			\inferrule*[left={\footnotesize$... ∈ \{ ... \}$},Right=Assump,vcenter=true,rightskip=-6em]
			{ }
			{p ∧ (q ∨ r) ⊢ p ∧ (q ∨ r)}
		}
		{p ∧ (q ∨ r) ⊢ q ∨ r}\\
		\inferrule*[Right=ImpIntro]
		{
			\inferrule*[Right=DisjIntroL]
			{
				\inferrule*[Right=ConjIntro]
				{
					\inferrule*[Left=ConjElimL]
					{
						\inferrule*[left={\footnotesize$... ∈ \{ ... \}$},Right=Assump,vcenter=true,rightskip=-6em]
						{ }
						{p ∧ (q ∨ r), q ⊢ p ∧ (q ∨ r)}
					}
					{p ∧ (q ∨ r), q ⊢ p}\\
					\inferrule*[left={\footnotesize$q ∈ \{ ... \}$},Right=Assump,vcenter=true,rightskip=-6em]
					{ }
					{p ∧ (q ∨ r), q ⊢ q}
				}
				{p ∧ (q ∨ r), q ⊢ p ∧ q}
			}
			{p ∧ (q ∨ r), q ⊢ (p ∧ q) ∨ (p ∧ r)}
		}
		{p ∧ (q ∨ r) ⊢ q → (p ∧ q) ∨ (p ∧ r)}
		%
	}
	{p ∧ (q ∨ r) ⊢ (p ∧ q) ∨ (p ∧ r)}
}
{⊢ p ∧ (q ∨ r) → (p ∧ q) ∨ (p ∧ r)}
$$
There is another branch next to the derivation of $p ∧ (q ∨ r) ⊢ q → (p ∧ q) ∨ (p ∧ r)$, namely the analogous derivation of $p ∧ (q ∨ r) ⊢ p → (p ∧ q) ∨ (p ∧ r)$. It is not part of the tree due to a lack of space.
\end{enumerate}
\end{landscape}
\end{solution}

\task[Programming in Coq]

   Consider the following \lstinline|option| data type that provides two kinds of values, either \lstinline|none| or \lstinline|some v| for some value \lstinline|v|. Like \lstinline|list|, it is parameterized over a type \lstinline|A|.
\begin{lstlisting}[numbers=none,xleftmargin=0em]
Inductive option (A: Type) : Type :=
| some : A -> option A
| none : option A.
\end{lstlisting}
Write the following programs, using the provided template file.
  %
  \begin{enumerate}[label=(\roman*)]
    \item \lstinline|pred_opt: nat -> option nat|\\
    This function returns the predecessor of a natural number, or \lstinline|none| if it does not exist.
		\item \lstinline|pred: nat -> nat|\\
    This function returns the predecessor of a natural number, or \lstinline|0| if it does not exist.
    \item \lstinline|minus: nat -> nat -> option nat|\\
    This function computes $x - y$ for two natural numbers. If the result is not a natural number, \lstinline|none| is returned.
    \item \lstinline|is_none: forall (X: Type), option X -> bool|\\
    This polymorphic function takes an optional value and returns \lstinline|true| if the value is \lstinline|none| and \lstinline|false| otherwise.
    \item \lstinline|head: forall (X: Type), list X -> option X|\\
    This function returns the first element of a list, or \lstinline|none| if the list is empty.
  \end{enumerate}

\task[Proving in Coq]
  Prove the following propositions in Coq, using the provided template file.

  \begin{enumerate}[label=(\roman*)]
		\item \lstinline|forall (X: Type) (x: X) (xs: list X), head (cons x xs) =|~\lstinline|some x|
		\item \lstinline|forall (n: nat), S (pred n) =|~\lstinline|n \/|~\lstinline|n|~\lstinline|= 0|
		\item \lstinline|forall (x y z: nat), (x + y) + z =|~\lstinline|x + (y + z)|
		\item \lstinline|forall (m n: nat), minus m n |~\lstinline|= none \/|~\lstinline|minus n m =|~\lstinline|none \/|~\lstinline|m =|~\lstinline|n|
  \end{enumerate}



\end{document}
