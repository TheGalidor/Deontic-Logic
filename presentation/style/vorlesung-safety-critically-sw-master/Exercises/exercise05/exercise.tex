\documentclass[language=en,sheet=5,prefix]{exercise}

\exerciseCourse{Development of Safety Critical Systems(Sichere Software)}
\exerciseAuthors{Martin Leucker, Grigory Markin, Felix Lange}
\exerciseTerm{winter term 18}
\exerciseDeadline{Sunday December 2, 2018 24:00}


\usepackage{hyperref}
\usepackage{utf8-math}
\usepackage{xspace}
\usepackage{tikz}
\usetikzlibrary{arrows,automata}


\begin{document}

CPAchecker is a tool for configurable software verification. 
The tool is available as a local installation, download from \url{https://cpachecker.sosy-lab.org}, 
as well as a cloud service available at
\url{https://cpachecker.appspot.com}. The cloud version of the tool has some limitations 
and can not be used for e.g. bit precise predicate analysis. The list of unavailable features 
can be found on the service page.


\task[Getting started with CPAchecker]

\begin{itemize}
\item For local installation, please, follow the instruction at 
\url{https://github.com/sosy-lab/cpachecker/blob/trunk/INSTALL.md}.
\textbf{It is highly recommend to use Linux for bit precise predicate analysis. If you use other OS, consider to run the tool inside e.g. a Virtualbox with Linux OS.}

\item Read a "Getting started" guide at \url{https://github.com/sosy-lab/cpachecker/blob/trunk/README.md}
\end{itemize}

%----------------------------------------------------------------------------------
%----------------------------------------------------------------------------------

\task[Program 1]

\begin{itemize}
\item Consider the following C program:
\begin{lstlisting}[language=C]
int x, y, z, w;

int main() {
  do {
    x = y;
    if (w != 1) {
      x++;
    }      
    z = w - 1;
  } while (x != y);

  if (z) {
  ERROR:
    return 1;
  }

  return 0;
}
\end{lstlisting}
\item Try to find out whether the ERROR location is reachable ?
\item Copy the file "exercise-non-precise.properties" and "program1.c" into the CPAchecker directory.
\item Verify the program using CPAchecker by executing the following command from the CPAchecker directory:
  \begin{lstlisting}[language=bash,numbers=none]
$ scripts/cpa.sh -config exercise-non-precise.properties program1.c
  \end{lstlisting}
\item Exemine the output located in the "CPAchecker/output" directory.
\item Change the condition at line $6$ to $w == 1$, run the tool again and check the report.
\end{itemize}


%----------------------------------------------------------------------------------
%----------------------------------------------------------------------------------

\task[Program 2]
\begin{itemize}
\item Consider the following C program:
\begin{lstlisting}[language=C]
int __VERIFIER_assert(int cond) {
  if (!(cond)) {
  ERROR:
    return 1;
  }
  return 0;
}

int glob_arr[] = {1, 2, 3, 4, 5};

int sum_arr(int *from, int *to) {
  int result = 0;
  while (from < to) {
    result += *(from++);
  }
  return result;
}

int main() {
  int sum = sum_arr(glob_arr, glob_arr + 5);
  return __VERIFIER_assert(sum == 15);
}
\end{lstlisting}
\item Copy the file "program2.c" into the CPAchecker directory.
\item Verify the program using the same configuration file as before.
\item Is the ERROR location rechable ? Are the results of the verification correct ?  
% \item Copy the file "exercise-precise.properties" into the CPAchecker directory.
% \item Verify the program using CPAchecker by executing the following command from the CPAchecker directory:
%   \begin{lstlisting}[language=bash,numbers=none]
% $ scripts/cpa.sh -config exercise-precise.properties program2.c
%   \end{lstlisting}
%   \textbf{Note: this configuration needs local installation in Linux.}
% \item Compare new and old results.
\end{itemize}

\end{document}
