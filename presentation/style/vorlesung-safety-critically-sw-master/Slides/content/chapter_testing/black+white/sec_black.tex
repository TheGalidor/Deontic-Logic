\section{Black-Box Testing}

% Computer-assisted Unit Testing

\begin{Frame}{Computer-assisted Unit Testing}
  Test cases to be executed by a computer\\
  can have several levels of automation:

  \begin{itemize}
    \item Test suite generated \alert{manually}\\
      (still the most common case)
    \item Test suite generated \alert{manually with some tool assistance }
    \item Test suite generated \alert{automatically}\\
      (specific areas, mostly research)
  \end{itemize}
  
  \xxx

  \begin{description}
    \item[Black-box testing] \alert{no access} to source code,\\
      test cases designed solely \alert{from specification}
    \item[White-box testing] \alert{access} to source code,\\
      test cases designed \alert{from specification and code}
  \end{description}
\end{Frame}

\begin{Frame}{Methodologies to Design a Test Suite}
  \inhead{Black-box Testing}

  \begin{itemize}
    \item Equivalence partitioning
    \item Boundary-value analysis
  \end{itemize}

  \xxx

  \inhead{White-box Testing}

  \begin{itemize}
    \item Control-flow based path coverage
    \item Data-flow based path coverage
  \end{itemize}
\end{Frame}

\begin{Frame}{Methodologies to Design a Test Suite}{Black-box Testing}
  \begin{definition}[Equivalence Partitioning]
    Partition the \alert{input domain} of a program into a finite number of
    \alert{equivalence classes} such that one can \alert{reasonably assume} that a test of a representative value of each class is equivalent to a test of any other value.
  \end{definition}

  \xxx

  \begin{definition}[Boundary-value Analysis]
    Choose test cases directly \enquote{on}, \enquote{above}, and \enquote{beneath} the edges of equivalence classes.
  \end{definition}

  \xxx

  Usually applied together, boundary-value analysis used to add new test cases after equivalence partitioning.s
\end{Frame}


\subsection{Equivalence Partitioning}

\begin{Frame}{Equivalence Partitioning}{I. Identify the equivalence classes}
  \begin{itemize}
    \item Take \alert{each input condition} and partition it into
    \item at least one group of \alert{valid equivalence classes} and
    \item at least one group of \alert{invalid equivalence classes}.
  \end{itemize}

  \xxx

  \begin{example}
    \inhead{Input condition}\newline
    \enquote{$x$ is an integer between 0 and 100.}\hfill\vspace{1ex}\linebreak
    

    \inhead{Equivalence classes}
    \begin{itemize}
      \item $0 \ge x \ge 100$ (valid inputs)
      \item $x < 0$ (invalid inputs)
      \item $x > 100$ (invalid inputs)
    \end{itemize}
  \end{example}
\end{Frame}

\begin{Frame}{Equivalence Partitioning}{II. Define the test cases}
  \begin{enumerate}
    \item Assign a unique number to each equivalence class.
    \item Until all equivalence classes have been covered by test cases, write a new test case covering \alert{as many of the uncovered valid equivalence classes as possible}.
    \item Until all equivalence classes have been covered by test cases, write a new test case that covers \alert{one and only one of the uncovered invalid equivalence classes}.
    \end{enumerate}

    \xxx

    \begin{alertblock}{Why?}
      \pause
      \begin{itemize}
        \item Positive examples should \alert{cover many} valid equivalence classes to make the tests more complicated.
        \item Negative examples should \alert{cover one} invalid equivalence class to identify a specific bug when failing.
      \end{itemize}
    \end{alertblock}
\end{Frame}

\begin{Frame}[fragile]{\lstinline-mylogin- Example}
  Given a user ID and a password the function \lstinline-mylogin- checks whether the user can login.

  \begin{lstlisting}[language=Java,gobble=4]
    boolean mylogin(String userid,
                    String password);
  \end{lstlisting}

  \begin{itemize}
    \item \lstinline-userid- must be an alphanumeric string of length between 1 and 32 characters containing no numerical character before an alphabetic character.
    \item \lstinline-password- must be a string between 8 and 64 characters without blanks, with at least one numerical or special character, and different from \lstinline-userid-.
  \end{itemize}
\end{Frame}

\begin{Frame}[fragile]{Equivalence Classes}{\lstinline-mylogin- Example}
	\begin{center}
  \footnotesize\begin{zebratabular}{p{3.3cm}p{2.1cm}p{3.3cm}}
  \headerrow input condition & Valid eq. class & invalid eq. class \\
  \lstinline-userid- is alphanumeric & yes (1) & no (2) \\
  $1 \le |\text{\lstinline-userid-}| \le 32$ & yes (3) & 
  $|\text{\lstinline-userid-}|< 1$ (4), \newline
  $|\text{\lstinline-userid-}|>32$ (5) \\
  no num. before alph. & yes (6) & 
  all char. numeric (7), \newline
  num. char. followed\newline by alph. char. (8)\\
  $8 \le |\text{\lstinline-password-}| \le 64$ & yes (9) & 
  $|\text{\lstinline-password-}|< 8$ (10), \newline
  $|\text{\lstinline-password-}|> 64$ (11)\\
  no blanks & yes (12) & no (13) \\
  one num. or sp. char. & 
  yes, num. (14), \newline
  yes, sp. (15) & no (16) \\
  diff. from \lstinline-userid- & yes (17) & no (18) \\
  \end{zebratabular}
  \end{center}
\end{Frame}

\begin{Frame}[fragile]{Test Cases}{\lstinline-mylogin- Example}
	\begin{center}
  \begin{zebratabular}{llllp{2cm}}
    \headerrow \# & userid &  password  & result & eq. classes \\
    1 & \texttt{leucker} & \texttt{leucker0} & accept & 1,3,6,9, 12,14,17 \\
    2 & \texttt{leucker} & \texttt{leucker!} & accept & 1,3,6,9, 12,15,17 \\
    3 & \texttt{leucker?} & \texttt{leucker!} & reject & 2 \\
    4 & \textit{empty} & \texttt{leucker!} & reject & 4 \\
    5 & \texttt{l$^{27}$eucker} & \texttt{leucker!} & reject & 5 \\
    6 & \texttt{7546} & \texttt{leucker!} & reject & 7 \\
    7 & \texttt{le5ucker} & \texttt{leucker!} & reject & 8 \\
    8 & \texttt{leucker} & \texttt{leucke0} & reject & 10 \\
    9 & \texttt{leucker} & \texttt{l$^{58}$eucker0} & reject & 11 \\
    10 & \texttt{leucker} & \lstinline[showspaces=true,identifierstyle={}]-leuc ker0- & reject & 13 \\
    11 & \texttt{leucker} & \texttt{lleucker} & reject & 16 \\
    12 & \texttt{leucker0} & \texttt{leucker0} & reject & 18 \\
  \end{zebratabular}
  \end{center} 
\end{Frame}

\subsection*{Boundary Values}

\begin{frame}{Boundary Values}
  \textcolor{maincolor}{Differs from equivalence partitioning in two respects:}

  \begin{itemize}
    \item Instead of selecting any element of an equivalence class, one or
      more elements are selected such that each \enquote{edge} of the equivalence class
      is the subject of a test.
    \item Test cases are also derived by considering the \alert{result space};
      i.e., output equivalence classes.
  \end{itemize} 
\end{frame}

\begin{Frame}{Boundary Values}{Some Guidelines}
  \begin{itemize}
    \item If an input condition specifies a range of values, write test cases for the ends of the range, and invalid-input test cases for situations just beyond the ends.
    \item If an output condition specifies a range of values, write test cases for the ends of the range, and see if there might be cases causing an output beyond the ends.
    \item If the input or output is an ordered set (array, string, list, table, \ldots) focus attention on the first and last elements of the set.
    \item Use your ingenuity to search for other boundary conditions.
  \end{itemize}
\end{Frame}

\begin{Frame}[allowframebreaks,fragile]{Test Cases From Boundary Values}{\lstinline-mylogin- Example}
  Write a test case in which all characters of \lstinline-userid- are alphanumeric.\hfill\vspace{1ex}\linebreak

  \begin{zebratabular}{llllp{2cm}}
    \headerrow \# & userid &  password  & result & eq. classes \\
    1 & \texttt{leucker} & \texttt{leucker0} & accept & 1,3,6,9, 12,14,17 \\
  \end{zebratabular}

  \framebreak

  Write test cases in which the first, the last, all characters of
  \lstinline-userid- are not alphanumeric.\hfill\vspace{1ex}\linebreak

  \begin{zebratabular}{llllp{2cm}}
    \headerrow \# & userid &  password  & result & eq. classes \\
    3 & \texttt{leucker?} & \texttt{leucker!} & reject & 2 \\
    13 & \texttt{!leucker} & \texttt{leucker0} & reject & \textcolor{alertedcolor}{new} \\
    14 & \texttt{!!!}  & \texttt{leucker0} & reject & \textcolor{alertedcolor}{new} \\
  \end{zebratabular}

  \framebreak

  Write test cases where \lstinline-userid- has length 0, 1, 32, and 33.\hfill\vspace{1ex}\linebreak

  \begin{zebratabular}{llllp{2cm}}
    \headerrow \# & userid &  password  & result & eq. classes \\
    4 & \textit{empty} & \texttt{leucker!} & reject & 4 \\
    5 & \texttt{l$^{27}$eucker} & \texttt{leucker!} & reject & 5 \\
    15 & \texttt{l} & \texttt{leucker0} & accept & \textcolor{alertedcolor}{new} \\
    16 & \texttt{l$^{26}$eucker} & \texttt{leucker0} & accept & \textcolor{alertedcolor}{new} \\
  \end{zebratabular}

  \framebreak

  Write a test case in which the first character of \lstinline-userid-
  is alphabetic and all others numeric.\hfill\vspace{1ex}\linebreak

  \begin{zebratabular}{llllp{2cm}}
    \headerrow \# & userid &  password  & result & eq. classes \\
    17 & \texttt{l1234} & \texttt{leucker0} & accept & \textcolor{alertedcolor}{new} \\
  \end{zebratabular}

  \framebreak

  Write test cases in which\\
  \ \ -- the first character of \lstinline-userid- is numerical,\\
  \ \ -- the last two characters are numeric-alphabetic,\\
  \ \ -- all characters are numeric.\hfill\vspace{1ex}\linebreak

  \begin{zebratabular}{llllp{2cm}}
    \headerrow \# & userid &  password  & result & eq. classes \\
    6 & \texttt{7546} & \texttt{leucker!} & reject & 7 \\
    18 & \texttt{1leucker} & \texttt{leucker0} & reject & \textcolor{alertedcolor}{new} \\
    19 & \texttt{leuc0r} & \texttt{leucker0} & accept & \textcolor{alertedcolor}{new} \\
  \end{zebratabular} 

  \framebreak

  Write test cases where \lstinline-password- has length 7, 8, 64, and 65.\hfill\vspace{1ex}\linebreak

  \begin{zebratabular}{llllp{2cm}}
    \headerrow \# & userid &  password  & result & eq. classes \\
    1 & \texttt{leucker} & \texttt{leucker0} & accept & 1,3,6,9, 12,14,17 \\
    8 & \texttt{leucker} & \texttt{leucke0} & reject & 10 \\
    9 & \texttt{leucker} & \texttt{l$^{58}$eucker0} & reject & 11 \\
    21 & \texttt{leucker} & \texttt{l$^{57}$eucker0} & accept & \textcolor{alertedcolor}{new} 
  \end{zebratabular}

  \framebreak

  Write a test case in which \lstinline-password- contains no blanks.\hfill\vspace{1ex}\linebreak

  \begin{zebratabular}{llllp{2cm}}
    \headerrow \# & userid &  password  & result & eq. classes \\
    1 & \texttt{leucker} & \texttt{leucker0} & accept & 1,3,6,9, 12,14,17 \\
  \end{zebratabular}

  \framebreak

  Write test cases in which the first, the last, all characters of
  \lstinline-password- are blanks.\hfill\vspace{1ex}\linebreak

  \begin{zebratabular}{llllp{2cm}}
    \headerrow \# & userid &  password  & result & eq. classes \\
    22 & \texttt{leucker} & \lstinline[showspaces=true,identifierstyle={}]- leucker- & reject & \textcolor{alertedcolor}{new} \\
    23 & \texttt{leucker} & \lstinline[showspaces=true,identifierstyle={}]-leucker - & reject & \textcolor{alertedcolor}{new} \\
    24 & \texttt{leucker} & \lstinline[showspaces=true,identifierstyle={}]-   - & reject & \textcolor{alertedcolor}{new} \\
  \end{zebratabular}

  \framebreak

  Write a test case in which just the first or the last character of
  \lstinline-password- is numeric.\hfill\vspace{1ex}\linebreak

  \begin{zebratabular}{llllp{2cm}}
    \headerrow \# & userid &  password  & result & eq. classes \\
    1 & \texttt{leucker} & \texttt{leucker0} & accept & 1,3,6,9, 12,14,17 \\
    25 & \texttt{leucker} & \texttt{0leucker} & reject & \textcolor{alertedcolor}{new} \\
  \end{zebratabular} 

  \framebreak

  Write a test case in which just the first or the last character 
  of \lstinline-password- is special.\hfill\vspace{1ex}\linebreak

  \begin{zebratabular}{llllp{2cm}}
    \headerrow \# & userid &  password  & result & eq. classes \\
    2 & \texttt{leucker} & \texttt{leucker!} & accept & 1,3,6,9, 12,15,17 \\
    26 & \texttt{leucker} & \texttt{!leucker} & reject & \textcolor{alertedcolor}{new} \\
  \end{zebratabular} 

  \framebreak

  Write a test case in which \lstinline-password-
  has no numeric or special characters.\hfill\vspace{1ex}\linebreak

  \begin{zebratabular}{llllp{2cm}}
    \headerrow \# & userid &  password  & result & eq. classes \\
    11 & \texttt{leucker} & \texttt{lleucker} & reject & 16 \\
  \end{zebratabular} 
\end{Frame}

\subsection{MTEST example}

\begin{Frame}{The MTEST example}{Specification of input I}
  MTEST is a program that grades multiple-choice examinations. The input is a file named OCR, which contains 80-character records. The first record is a title; the contents of this record are used as a title on each output report. The next set of records describes the correct answers on the exam. Each record contains a 2 as the last character. In the first record of this set, the number of questions is listed in columns 1--3 (a value of 1--999). Columns 10--59 contain the correct answers for questions 1--50 (any character is valid as an answer). Subsequent records contain, in columns 10--59, the correct answers for questions 51--100, 101--150, and so on.
\end{Frame}

\begin{Frame}{The MTEST example}{Specification of input II}
  The third set of records describes the answers of each student; each record contains a 3 in column 80. For each student, the first record contains the student's name or number in columns 1--9 (any characters); columns 10--59 contain the student's answers for questions 1--50. If the test has more than 50 questions, subsequent records for the student contain answers 51--100, 101--150, and so on, in columns 10--59. The maximum number of students is 200. 
\end{Frame}

\begin{Frame}{The MTEST example}{Specification of output}
  The four output records are 
  \begin{itemize}
    \item[(1)] a report, sorted by student identifier, showing each student's grade (percentage of answers correct) and rank; 
    \item[(2)] a similar report, but sorted by grade;  
    \item[(3)] a report indicating the mean, median, and standard deviation of the grades; and 
    \item[(4)] a report, ordered by question number, showing the percentage of students answering each questions correctly.
  \end{itemize}
\end{Frame}

\begin{Frame}{Test cases from input conditions I}
  \begin{itemize}
    \item[1.] Empty input file
    \item[2.] Missing title record
    \item[3.] 1-character title
    \item[4.] 80-character title
    \item[5.] 0-question exam
    \item[6.] 1-question exam
    \item[7.] 50-question exam
    \item[8.] 51-question exam
    \item[9.] 999-question exam
    \item[10.] Number-of-questions field has non-numeric value
  \end{itemize}
\end{Frame}

\begin{Frame}{Test cases from input conditions II}
  \begin{itemize}
    \item[11.] No correct-answer records after title record
    \item[12.] One too many correct-answer records
    \item[13.] One too few correct-answer records
    \item[14.] 0 students
    \item[15.] 1 student
    \item[16.] 200 students
    \item[17.] 201 students
  \end{itemize}
\end{Frame}

\begin{Frame}{Test cases from input conditions III}
  \begin{itemize}
    \item[18.] A student has one answer record, but there are two correct-answer records.
    \item[19.] The above student is the first student in the file.
    \item[20.] The above student is the last student in the file.
    \item[21.] A student has two answer records, but there is just one correct-answer record.
    \item[22.] The above student is the first student in the file.
    \item[23.] The above student is the last student in the file.
  \end{itemize}
\end{Frame}

\begin{Frame}{Test cases from output conditions}{Reports 1 and 2}
  \begin{itemize}
    \item 0, 1, 200  students (same as test 14,15,16)
    \item[24.] All students receive the same grade.
    \item[25.] All students receive a different grade.
    \item[26.] Some, but not all, students receive the same grade (to see if ranks are computed correctly).
    \item[27.] A student receives a grade of 0.
    \item[28.] A student receives a grade of 100.
    \item[29.] A student has the lowest possible identifier value (to check the sort).
    \item[30.] A student has the highest possible identifier value.
    \item[31.] The number of students is such that the report is just large enough to fit on one page (to see if an extraneous page is printed).
    \item[32.] The number of students is such that all students but one fit on one page
  \end{itemize}
\end{Frame}

\begin{Frame}{Test cases from output conditions}{Report 3}
  \begin{itemize}
    \item[33.] The mean is at its maximum (all students have a perfect score)
    \item[34.] The mean is 0 (all students receive a grade of 0)
    \item[35.] The standard deviation is at its maximum (one student receives a 0 and the other receives a 100).
    \item[36.] The standard deviation is 0 (all students receive the same grade).
  \end{itemize}
\end{Frame}

\begin{Frame}{Test cases from output conditions}{Report 4}
  \begin{itemize}
    \item[37.] All students answer question 1 correctly.
    \item[38.] All students answer question 1 incorrectly.
    \item[39.] All students answer the last question correctly.
    \item[40.] All students answer the last question incorrectly.
    \item[41.] The number of questions is such that the report is just large enough to fit on one page.
    \item[42.] The number of questions is such that all questions but one fit on one page.
  \end{itemize}
\end{Frame}