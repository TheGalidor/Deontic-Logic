%!TEX root = ../../../main.tex

\section{Bounded Model Checking}

\begin{frame}{Bounded Model Checking (1)}

A technique to falsify invariants
\begin{itemize}
  \itemsep1em
  \item best suited for ``bug finding''
  \item checks the existence of counterexamples to the property
\end{itemize}

\bigskip
Based on unrolling the transition system up to some bound
\begin{itemize}
  \itemsep1em  
  \item considers execution traces of the length $k$
\end{itemize}
\end{frame}

% ---------------------------------------------------------------------

\begin{frame}{Bounded Model Checking (2)}
Encoding of the transition system in any logic (or fragment of a logic) $\logic$
with decidable entailment
\begin{itemize}
  \itemsep1em
  \item e.g. quantifier-free real (or linear integer) arithmetic with arrays and   
  uninterpreted functions  
  \item model-checking problem is encoded as satisfiability problem in $\logic$
\end{itemize}

\bigskip
Bounded Model Checking is unsound in general:
\begin{itemize}
  \itemsep1em
  \item under-approximation and hence:
  \item not finding bugs does not mean its absense in the system
\end{itemize}
\end{frame}

% ---------------------------------------------------------------------

\begin{frame}{Algorithm}
\begin{enumerate}
\itemsep1.2em
\item Constructs a formula in logic $\logic$ that is satisfiable iff there exists a
length-$k$ counterexample:
$$I(s_0) \wedge \bigwedge_{0 \leq i < k} T(s_i, s_{i+1}) \wedge \neg \phi(s_k)$$

\item If the formula is satisfiable, there exists at least one
$k$-length counterexample
\begin{itemize}
  \item Each variable assignment represents a counterexample
\end{itemize}

\item If no counterexample is found (the formula is unsatisfiable), the method
increases $k$ until
\begin{itemize}
  \item a counterexample is found
  \item the search becomes intractable, or
  \item $k$ reaches a certain bound
\end{itemize}
\end{enumerate}
\end{frame}

% ---------------------------------------------------------------------

\begin{frame}{Example}

\begin{center}
  \begin{tikzpicture}[>=stealth',shorten >=1pt,auto,node distance=2cm, initial
  text={}] 
    \node[initial,state, onslide=<2->{highlight}] (q0)
      {$q_0$};
      \node[state, onslide=<3->{highlight}] (q1) [right of = q0]    
        {$q_1$};        
      \node[state, onslide=<4->{highlight}] (q2) [right of = q1]
        {$q_2$};
    \node[state, bad] (q3) [right of = q2]
      {$q_3$};
    
    
        
      \path[->]   (q0)  edge [loop above] node {} (q0)
            (q0)  edge        node {} (q1)
            (q1)  edge [loop above] node {} (q1)  
                edge        node {} (q2)                    
            (q3)  edge [loop above] node {} (q3)  
    ;
    
    \only<6->{
      \path[->] (q2) edge [] node {} (q3);
    }
              
  \end{tikzpicture}
\end{center}

\bigskip
\uncover<2->{
\alt<6>{
\begin{itemize}
  \item[] \underline{$k=3$}: \quad\small  $I(s_0) \wedge T(s_0, s_1)
  \wedge T(s_1, s_2) \wedge T(s_2, s_3) \wedge \neg \phi(s_3)$
  \begin{itemize}
    \item[] \hl{Counterexample}: $q_0 \rightarrow q_1 \rightarrow q_2 \rightarrow q_3$
  \end{itemize}
\end{itemize}
}{
\begin{itemize}
  \item[] \underline{$k=0$}:  \quad $I(s_0) \wedge \neg \phi(s_0)$
  \begin{itemize}
    \item[] $q_0$
  \end{itemize}
\end{itemize}
}}

\uncover<3-5>{
\begin{itemize}
  \item[] \underline{$k=1$}:  \quad $I(s_0) \wedge T(s_0, s_1) \wedge \neg
  \phi(s_1)$
  \begin{itemize}
    \item[] $q_0 \rightarrow q_0$; \quad $q_0 \rightarrow q_1$    
  \end{itemize}
\end{itemize}
}

\uncover<4-5>{
\begin{itemize}
  \item[] \underline{$k=2$}:  \quad $I(s_0) \wedge T(s_0, s_1) \wedge T(s_1, s_2)
  \wedge \neg \phi(s_2)$
  \begin{itemize}
    \item[] $q_0 \rightarrow q_0 \rightarrow q_0$; \quad $q_0 \rightarrow q_0
    \rightarrow q_1$; \quad $q_0 \rightarrow q_1 \rightarrow q_1$; \quad $q_0
    \rightarrow q_1 \rightarrow q_2$
  \end{itemize}
\end{itemize}
}

\uncover<5>{
\begin{itemize}
  \item[] \underline{$k=n$}:  \quad $I(s_0) \wedge T(s_0, s_1) \wedge \ldots
  \wedge T(s_{n-1}, s_n) \wedge \neg \phi(s_n)$
  \begin{itemize}
    \item[] $\underbrace{q_0 \leadsto \ldots \leadsto q_0}_n$; \quad 
    $\underbrace{q_0 \leadsto \ldots \leadsto q_1}_n$; \quad 
    $\underbrace{q_0 \leadsto \ldots \leadsto q_2}_n$; \quad \ldots
  \end{itemize}
\end{itemize}
}
\end{frame}

% ---------------------------------------------------------------------

\begin{frame}{Bounded Model Checking}

The original BMC algorithm, although complete for finite-state systems, is
limited in practice to \hl{falsification}

\bigskip
BMC can prove that an invariant $\phi$ holds on a system $\kripke$ if a bound $k$
is known such that:
\begin{itemize}
  \item if no counterexample of length $k$ is found, $\kripke \satl \phi$
\end{itemize}

\bigskip
The optimum value of $k$ is usually very expensive to obtain
\begin{itemize}
  \item Finding it is at least as hard as the model-checking problem itself
\end{itemize}

\bigskip
Unrolling of the transition system appears to be a ``bootleneck'' of the method
\end{frame}

% ---------------------------------------------------------------------

% \begin{frame}{Bounded Model Checking}

% \begin{center}
%   \begin{tikzpicture}[>=stealth',shorten >=1pt,auto,node distance=2cm, initial
%   text={}] 
%     \node[initial,state] (q0)
%       {$q_0$};
%       \node[state] (q1) [right of = q0]   
%         {$q_1$};        
%       \node[state] (q2) [right of = q1]
%         {$q_2$};
%     \node[state, dashed] (q3) [right of = q2]
%       {$q_3$};
    
    
        
%       \path[->]   (q0)  edge [loop above] node {} (q0)
%             (q0)  edge        node {} (q1)
%             (q1)  edge [loop above] node {} (q1)  
%                 edge        node {} (q2)                    
%             (q3)  edge [loop above] node {} (q3)  
%     ;   
    
%     \only<3->{
%       \path[->] (q2) edge [] node {} (q3);
%     }
              
%   \end{tikzpicture}
% \end{center}

% \setbeamercovered{transparent}

% \uncover<1-2>{
% Can we prove that the bad state $q_3$ is unreachable without unrolling the
% transition relation ?
% \uncover<2->{\textcolor{green}{Yes}}
% }

% \slidepart{3}{
% Can we find a counterexample in more efficient way than using BMC ?
% \uncover<4->{\textcolor{brown}{Maybe}} 
% }

% \slidepart{4}{
% \begin{itemize}
%   \item<4-5> Interpolation-based model checking  
%   \item<4> Backward reachability
%   \item<4-5> Temporal induction ($k$-induction + enhancements)
%   \item<4-5> Inductive Generalization of Counterexamples to Induction
% \end{itemize}
% }

% \setbeamercovered{invisible}

% \end{frame}